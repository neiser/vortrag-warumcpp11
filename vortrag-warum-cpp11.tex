\pdfminorversion=4
\documentclass[t,ngerman,usepdftitle=false]{beamer}
\usepackage[utf8]{inputenc}
\usepackage[T1]{fontenc}
\usepackage{babel}
\usepackage{listings}
\usepackage{csquotes}

\lstset{language=[11]C++, numbers=left, numberstyle=\tiny, stepnumber=1, numbersep=5pt,
   basicstyle={\firalining\ttfamily}, moredelim=**[is][\color{red}]{@}{@},
 commentstyle=\color{gray}, frame=single, framerule={0.8pt}, columns=fullflexible,
literate=={=}{a}1}
\lstnewenvironment{cpplisting}{\lstset{basicstyle={\firalining\small},columns=fixed}}{}

%\usepackage[default]{comicneue}

\usepackage[sfdefault,scaled=1,book]{FiraSans}
\usepackage[lining]{FiraMono}

\usecolortheme{rose,seahorse}
\useinnertheme{rounded,circles}
\setbeamertemplate{navigation symbols}{}
\setbeamertemplate{footline}{%
  \raisebox{5pt}{\makebox[\paperwidth]{\hfill\makebox[10pt]{\scriptsize\insertframenumber}}}}

\newcommand\Cpp{%
  C\kern-.03em\raise.15ex\hbox{++}%
  \spacefactor1000}

\author{Andreas Neiser}
\title{Warum \Cpp{}11?}
\subtitle{Nur weil GCC6 jetzt \lstinline|-std=gnu++14| default macht?!}
\date{\today}
\hypersetup{pdftitle={Warum C++11?}}
\begin{document}

\begin{frame}[fragile]
  
    \maketitle
  
  \begin{center}
    \lstinline|!ErrorHasOccured() ??!??! HandleError();|
  \end{center}
\end{frame}


\begin{frame}[fragile]
  \frametitle{AAA -- Almost always \lstinline!auto! -- Fast immer \lstinline!auto! }
  Statt Typ hinzuschreiben, einfach \lstinline!auto!. \\ Der Compiler findet's schon raus, hoffentlich\ldots
  
\begin{cpplisting}
typedef vector<int> nums_t; 
static const int t[] = {1,2,3};
nums_t a(t,t+sizeof(t)/sizeof(t[0])); 
for(nums_t::iterator i=a.begin();  i!=a.end(); ++i)
  *i *= 2;
\end{cpplisting}

Wo weiß der Compiler eigentlich eh schon, \\ was für ein Typ die Variable hat?

\pause 
\lstset{firstnumber=6}
\begin{cpplisting}
static const int t[] = {1,2,3};
vector<int> a(t,t+sizeof(t)/sizeof(t[0])); 
for(@auto@ i=a.begin();  i != a.end(); ++i)
  *i *= 2;
\end{cpplisting}

\pause 
\begin{block}{}
  \centering
  \lstinline!auto! macht Code schlanker und allgemeiner\\ \small (und man kann über wichtigere Dinge nachdenken als \lstinline|typedef|s) 
\end{block}
  
\end{frame}

\begin{frame}[fragile]
  \frametitle{Geschweifte Klammern und \lstinline|initializer_list|}
\begin{cpplisting}
vector<int> a{1,2,3}; // oh, wie einfach!
for(auto i=a.begin();  i != a.end(); ++i)
  *i *= 2;
\end{cpplisting}

\pause
Genauso geht:
\lstset{firstnumber=4}
\begin{cpplisting}
struct my_t { 
  int A; int B;
  my_t(int a, int b) : A(a), B(b) {}
};
vector<my_t> a{{1,2},{3,4}};  
\end{cpplisting}

\pause
Aber {\color{red}Vorsicht} bei:\\
\lstset{firstnumber=9}
\begin{cpplisting}
vector<int> a(5);
vector<int> b{5};
\end{cpplisting}

\pause 
\begin{block}{}
  \centering
  \lstinline!{...}! macht das Initialisieren \enquote{natürlich hübsch}
\end{block}

\end{frame}


\begin{frame}[fragile]
  \frametitle{for-ranged loops und nochmal \lstinline!auto!}
\begin{cpplisting}
vector<int> a{1,2,3};
for(auto i=a.begin();  i != a.end(); ++i)
  *i *= 2;
\end{cpplisting}

\pause
\ldots{}kann man auch so schreiben:
\lstset{firstnumber=4}
\begin{cpplisting}
vector<int> a{1,2,3};
for(auto& v : a)
  v *= 2;
\end{cpplisting}

\pause
Hm, aber was ist mit:
\lstset{firstnumber=7}
\begin{cpplisting}
@const@ vector<int> a{1,2,3};
for(auto& v : a)
  v *= 2;
\end{cpplisting}

\pause
\begin{block}{}
  \centering
  for-ranged loops machen STL container endlich benutzbar,
  und \lstinline!auto! ist schlau genug für \lstinline!const!-correctness (und mehr)
\end{block}
  
\end{frame}

\begin{frame}[fragile]
  \frametitle{Zero-overhead byte-packing}
  \begin{cpplisting}
auto packed = Make(0,@1@,@1@,0,
                   @1@,0,0,@1@,	
                   @1@,0,0,@1@,
                   0,@1@,@1@,0); // = {0x69, 0x96}
  \end{cpplisting}

\pause
Eine erste Idee, schonmal mit variadic templates:
\lstset{firstnumber=5}
\begin{cpplisting}
using byte_t = unsigned char;
template<typename... Bools>
auto Make(Bools... bools) {
  const vector<bool> b{static_cast<bool>(bools)...};
  vector<byte_t> a(b.size()/8);
  for(auto i=0u; i<a.size(); ++i)
    a[a.size()-i-1] = (b[8*i+7]<<7) | (b[8*i+6]<<6) 
                    | (b[8*i+5]<<5) | (b[8*i+4]<<4) 
                    | (b[8*i+3]<<3) | (b[8*i+2]<<2) 
                    | (b[8*i+1]<<1) |  b[8*i+0];
  return a;
}
\end{cpplisting}
  
\end{frame}

\begin{frame}[fragile]
  \frametitle{Zero-overhead byte-packing II}
  
Erstmal: 
\begin{cpplisting}
template<typename... Bools>
auto Make(Bools... bools) {
  constexpr auto nBools = sizeof...(bools);
  static_assert(nBools % 8 == 0, "nBools wrong");
  auto a = array<byte_t, nBools/8>();
  Fill(a, nBools/8, bools...);
  return a;
}
\end{cpplisting}

\lstinline|Fill(...)| wird dem Compiler nun sagen, \\ wie \lstinline|a| zu füllen ist\ldots

\end{frame}

\begin{frame}[fragile]
  \frametitle{Zero-overhead byte-packing III}

\ldots{}und zwar mit rekursiven function calls:
\begin{cpplisting}
template<size_t N, typename... Bools>
void Fill(array<byte_t, N>& a,
          size_t i,
          bool b7, bool b6, bool b5, bool b4,
          bool b3, bool b2, bool b1, bool b0,
          Bools... bools) {
  a[N-i] = (b7<<7) | (b6<<6) /* | ... */ | b0;
  Fill(a, i-1, bools...);
}

template<size_t N>
void Fill(array<byte_t, N>&, size_t) {}
\end{cpplisting}

\pause
\begin{block}{}
  \centering Auf zur Livedemo mit \url{https://godbolt.org/g/57y21s}\\
  {\small mehr bei Jason Turner's CppCon2016 talk: \url{https://youtu.be/zBkNBP00wJE}}
\end{block}
    
\end{frame}


\begin{frame}[fragile]
  \frametitle{Nie wieder schreckliches raw pointer management}
  Warum will man sowas nicht:
  \begin{cpplisting}
auto p = new int;
DoSomething(p);
delete p;
  \end{cpplisting}
  
  \pause Besser, aber noch nicht gut:
  \lstset{firstnumber=4}
  \begin{cpplisting}
struct owner_t {
  int* p;
  owner_t(int* p_) : p(p_) {}
  ~owner_t() { delete p; }    
};  

{ // some scope
  owner_t o(new int);
  DoSomething(o.p);
}
  \end{cpplisting}
  
\end{frame}

\begin{frame}[fragile]
  \frametitle{Nie wieder schreckliches raw pointer management II}
  Was passiert bei Kopien von \lstinline|owner_t|:
  \begin{cpplisting}
{ // some scope
  owner_t o(new int);
  auto o_copy = o;
  DoSomething(o.p);
  // uh uh, double-free *zonk*
}
\end{cpplisting}
\pause
\begin{block}{}
  \centering
  Ownership sollte nicht kopiert werden können!   
\end{block}

\pause
Was ist dann aber mit:
\lstset{firstnumber=7}
\begin{cpplisting}
vector<owner_t> ptrs{new int()}; // oh no, again
\end{cpplisting}

\pause
\begin{block}{}
  \centering
  Man kann STL Container so nicht mehr verwenden!\\
  {\small Ähm, nein\ldots}   
\end{block}

  
\end{frame}

\begin{frame}[fragile]
  \frametitle{Move semantics mit \lstinline|unique_ptr<T>|}
  Die Rettung ist \lstinline|unique_ptr<int>| statt \lstinline|owner_t|
\begin{cpplisting}
{ // some scope
  auto o = make_unique<int>(); // factory
  DoSomethingWithRef(*o);
  DoSomething(move(o));
  // o invalid here, as DoSomething took ownership
}
\end{cpplisting}

\pause
Aber es ist nicht alles Sonnenschein:
\lstset{firstnumber=7}
\begin{cpplisting}
using p_t = unique_ptr<int>;
p_t in[] = {make_unique<int>(1),
            make_unique<int>(2)};
const list<p_t> ptrs(make_move_iterator(begin(in)),
                     make_move_iterator(end(in)));
for(const auto& p : ptrs)
  *p *= 2; // constness?! 
\end{cpplisting} 
  
  
\end{frame}


\begin{frame}[fragile]
  \frametitle{Endlich Ende}
  
  Was war:
  \begin{itemize}
    \item \lstinline|auto| geht einfach immer
    \item \lstinline|{...}| als Initialisierungsliste
    \item Variadic templates mit zero-overhead dank \lstinline|-O3|
    \item Object ownership management
  \end{itemize}
  
  Was ist:
  \begin{itemize}
    \item Scott Meyers, Effective Modern C++, 2014
    \item Bjarne Stroustrup, The \Cpp{} Programming Language, 2013
    \item CppCon auf YouTube
  \end{itemize}
  
  Was wird (vielleicht):
  \begin{itemize}
    \item Concepts für \lstinline|template|
    \item Structured bindings für \lstinline|tuple|
  \end{itemize}
  
\end{frame}

\end{document}